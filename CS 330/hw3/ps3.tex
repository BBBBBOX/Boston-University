\documentclass[12pt]{article}
\setlength{\textheight}{9.3in}
\setlength{\textwidth}{7.1in}
\setlength{\evensidemargin}{-0.2in}
\setlength{\oddsidemargin}{-0.2in}
\setlength{\headsep}{30pt}
\setlength{\topmargin}{-0.6in}
\usepackage{amsthm}
\usepackage{hyperref}
\newtheorem{theorem}{Theorem}
\theoremstyle{definition}
\newtheorem{definition}{Definition}

\newcommand{\addmedskip}{\addvspace{\medskipamount}}

\newcommand{\addbigskip}{\addvspace{\bigskipamount}}

\pagestyle{myheadings}
\markboth{BU CAS CS 330.  Fall 2015.}{BU CAS CS 330.  Fall 2015.} 
\newcounter{problemnum}
\setcounter{problemnum}{0}
\newenvironment{problem}
     {\addbigskip \setcounter{partnum}{0}
      \noindent\stepcounter{problemnum}\textbf{Problem
                                             \arabic{problemnum}.\ }}
     {\par\addbigskip}

\newcounter{partnum}
\setcounter{partnum}{0}
\newenvironment{ppart}
     {\addmedskip
      \noindent\stepcounter{partnum}\textbf{(\alph{partnum})}\ }
     {\par\addbigskip}



\begin{document}
\begin{center}
\Large{\textbf{CAS CS 330.  Problem Set 3}}\\
\smallskip
\large{\textbf{Due by 11:59pm on Wednesday, February 11\\ electronically via websubmit}}
\end{center}

\medskip

\begin{problem} (40 points)
You are given a set $S$ of $n$ intervals $(s_i,f_i)$, $1\le i\le n$.
You want to find a minimum subset $B^*\subseteq S$ such that for every
interval in $S$, there is a interval in $B^*$ that overlaps it.
(Motivating example: you have $n$  workers working
different shifts, one shift each.  You want to organize a
supervisory board as a subset of the workers
so that every worker's shift overlaps with at least
one board member's shift.)  Design an efficient algorithm
to do so, prove that it is correct, and analyze its running time.  Hint:  be greedy.
\end{problem}

\begin{problem} (20 points)
Consider the following generalization of the shortest path problem. Not only edges have costs, but also vertices: the cost of a vertex $u$ will be denoted by $w(u)$. The cost $w(P)$ of a path $P$ consisting of vertices $u_0, u_1, \dots , u_k$ is defined as the sums of the vertex and edge costs, except the starting vertex: $w(P)=w(u_0,u_1)+w(u_1)+w(u_1,u_2)+w(u_2)+\dots+w(u_{k-1},u_k)+w(u_k)$.(We did not include the cost of the starting vertex $u_0$ in order to make the cost of two paths $P_1$ and $P_2$ additive in the case when $P_2$ starts where $P_1$ ends.) Now consider the problem:
given a directed graph G with nonnegative vertex and edge costs and vertices $s, t$, find the lowest-cost path from $s$ to $t$. Show an algorithm solving this problem, prove its correctness, and analyze its running time. (Hint: Find a way to reduce this problem to the one in which there are only edge costs.)
\end{problem}

\begin{problem}  (40 points)
You are given a set of positive values $p_1< \dots< p_k$ and a set of
negative values $n_1> \dots > n_k$.  You are driving a car that starts at
zero and drives up and down along the real line.  It takes you unit time to
cover unit distance, and eventually you need to get to each point $p_1,
\dots, p_k$ and $n_1, \dots, n_k$.  Clearly, the solution that takes the
least amount of time is to first drive all the way out to $p_k$ or $n_k$,
whichever has the lower absolute value, and then come back to $n_k$ or
$p_k$, respectively.  However, you are not interested in spending the least
amount of total time.  Rather, because your job is drop off $2k$ senior
executives (one at each point), and senior executives charge by the minute,
you want want to minimize the total productivity lost.  More precisely,
assume that executive who needs to be dropped off at point $x$ (where
$x\in\{p_1, \dots, p_k, n_1, \dots n_k\}$) charges you amount $a_x$ for
each unit of time until dropped off.  In that case, you may want to turn
around and may pass zero multiple times, to drop off more expensive
executives sooner, even though your total time may increase.  Note:
executives are limber and jump out quickly, so no need to account for time
spent slowing down and stopping---when you pass a point $x$ on the way to
point $y$, you can drop off the executive for point $x$ without slowing
yourself down.
                                                                                
Describe an algorithm to minimize your total cost by converting this
problem to a shortest-path problem in a graph with $\Theta(k^2)$ nodes.
Prove that your algorithm is correct and analyze its running time. (Hint: your main challenge
is to figure out how to construct this graph. Each node
in your graph should correspond to a possible state of your world. The state of your world
starts out with no one dropped off and you at point 0; it evolves as you move; and it ends
with everyone dropped off and you at either $n_k$ or $p_k$. Once you figure out what the nodes are,
you also have to figure out what the edges are.)
\end{problem}
\end{document}
