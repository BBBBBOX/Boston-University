\documentclass[12pt]{article}
\setlength{\textheight}{9.3in}
\setlength{\textwidth}{7.1in}
\setlength{\evensidemargin}{-0.2in}
\setlength{\oddsidemargin}{-0.2in}
\setlength{\headsep}{30pt}
\setlength{\topmargin}{-0.6in}
\usepackage{amsthm}
\usepackage{hyperref}
\newtheorem{theorem}{Theorem}
\theoremstyle{definition}
\newtheorem{definition}{Definition}

\newcommand{\addmedskip}{\addvspace{\medskipamount}}

\newcommand{\addbigskip}{\addvspace{\bigskipamount}}

\pagestyle{myheadings}
\markboth{BU CAS CS 330.  Fall 2015.}{BU CAS CS 330.  Fall 2015.} 
\newcounter{problemnum}
\setcounter{problemnum}{0}
\newenvironment{problem}
     {\addbigskip \setcounter{partnum}{0}
      \noindent\stepcounter{problemnum}\textbf{Problem
                                             \arabic{problemnum}.\ }}
     {\par\addbigskip}

\newcounter{partnum}
\setcounter{partnum}{0}
\newenvironment{ppart}
     {\addmedskip
      \noindent\stepcounter{partnum}\textbf{(\alph{partnum})}\ }
     {\par\addbigskip}



\begin{document}
\begin{center}
\Large{\textbf{CAS CS 330.  Problem Set 1}}\\
\smallskip
\large{\textbf{Due by 11:59pm on Wednesday, January 28, electronically\\ (instructions to be posted on Piazza)}}
\end{center}

\medskip

\noindent\textbf{Note}: I encourage you to type up your problem sets; else, write them up and scan them in. Typing up your work allows to you revise it more easily, which means you are more likely to write clear and sound solutions. I specifically encourage you to use \LaTeX\ because it allows to focus on contents while it takes care of the form. This page \url{http://latex-project.org/ftp.html} provides distributions for the most common operating systems. Irecommend a good GUI front end with a built-in viewer that syncs between source and output (typically Ctrl-Click or Cmd-Click takes you back and forth).

\medskip
\noindent
\textbf{Reading:}  Read the \href{http://www.cs.bu.edu/~reyzin/teaching/s15cs330/syllabus.html}{course syllabus}, and pay particular attention to the collaboration policy.  Read sections 1.2, 2.1, and 2.2 of the book.

\medskip

\noindent
\textbf{Sign up for Piazza if you haven't yet}

\medskip

\begin{problem} (20 points)
Let's modify the stable matching problem to be gender-free: there are simply $2n$ people, each ranks $2n-1$ others, and we want to pair them into $n$ pairs. Show that a stable matching may not always exist.
(Note: this problem is quite timely, as the US Supreme Court agreed last week to consider whether the US Constitution permits states limit marriage to opposite-gender couples only; don't wait for them, as their ruling is expected after the problem set is due.)
\end{problem}

\begin{problem} (30 points)
Chapter 1, problem 7. Hint: this is the stable matching problem wearing a different outfit.
\end{problem}

\begin{problem} (26 points)
Read chapters 2.1 and 2.2 as review material, and the review of asymptotic notation available on Piazza. Rank the following functions by order of growth; that is, find an
arrangement $g_1, g_2, \ldots, g_{20}$ of the functions satisfying
$g_1=\Omega(g_2)$, $g_2=\Omega(g_3)$, \ldots, $g_{19}=\Omega(g_{20})$.
Partition your list into equivalence classes (e.g., by underlining or circling
equivalent functions) such that $f(n)$ and $g(n)$ are in the same class if
and only if $f(n)=\Theta(g(n))$.  (Note: below, $\lg n$ means $\log_2 n$).
\[
\renewcommand{\arraystretch}{1.25}
\begin{array}{cccccc}

n^{100} & (\sqrt{2})^{\lg n} & n^2 & \sqrt{n}  & 2^{2^{n+1}}\\
n^3 & \lg^2 n & \lg(n!) & 2^{2^n} & n^{1/\lg n}\\
\lg n & 2^{100 \lg{n}} & n\cdot 2^n & n^{\lg\lg n} & \log_3 5n \\
{n \choose 2} + n \lg n & 4^{\lg n} &
 n &  n\lg
n & 2^{\lg^{1.001} n}\\
\end{array}
\renewcommand{\arraystretch}{1}
\]
\end{problem}

\begin{problem}
(24 points)
For each of the following relationships, find two functions $f(n)$ and
$g(n)$ that satisfy it.  If no such $f$ and $g$ exist, write ``NONE''
and briefly explain why no such functions exist.

\begin{ppart}
$f(n) = o(g(n))$ and $f(n) \neq \Theta(g(n))$.
\end{ppart}

\begin{ppart}
 $f(n) = \Theta(g(n))$ and $f(n) = o(g(n))$.
\end{ppart}

\begin{ppart}
 $f(n) = \Theta(g(n))$ and $f(n) \neq O(g(n))$.
\end{ppart}

\begin{ppart}
 $f(n) = \Omega(g(n))$ and $f(n) \neq O(g(n))$.
\end{ppart}

\begin{ppart}
 $f(n) = \Omega(g(n))$ and $f(n) \neq o(g(n))$.
\end{ppart}


\begin{ppart}
 $f(n) = \omega(g(n))$ and $g(n) \neq o(f(n))$.
\end{ppart}
\end{problem}




\end{document}
